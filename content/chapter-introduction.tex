% !TEX root = ../thesis-example.tex
%

\chapter{Introduction}
\label{sec:intro}

\section{Motivation and Problem Statement (French version)}

L'ORSA (Own Risk Solvency and Assessment) est un ensemble de règles définies par la directive européenne Solvabilit\'e II, destiné à servir d'outil d'aide à la décision et d'analyse stratégique des risques. Dans le cadre de l'ORSA, les compagnies d'assurance doivent évaluer leur solvabilité future de façon continue et prospective. Pour ce faire, elles doivent notamment obtenir des projections probabilistes de leur bilan (actif et passif) sur un certain horizon temporel.   

Dans ce travail de thèse, nous nous focalisons sur l'actif du bilan. Nous traitons plus précisément des taux d'intérêts; de la construction, de l'extrapolation, et des prévisions envisagées dans le futur pour la courbe de taux d'intérêt. Les techniques présentées sont essentiellement basées sur des méthodes d'apprentissage statistique.    

Nous parlerons généralement dans le texte de \textit{courbe de taux}, mais il s'agit en fait  de courbes de facteurs d'actualisation sans risque de défaut de contrepartie. Le risque de défaut de contrepartie n'est pas explicitement traité, mais des techniques similaires à celles développées ici, peuvent être transposées à la construction de courbe de taux incorporant un risque de défaut de contrepartie.

Nous présentons dans un premier temps, une famille de courbes de facteurs d'actualisation basées sur les modèles de taux (courts) dits d'absence d'opportunité d'arbitrage ou exogènes. À une date donnée, la courbe d'actualisation est construite à l'aide de produits dérivés de taux, dont les cotations sont disponibles sur les marchés financiers pour un nombre donné de maturités. La régularité de la courbe obtenue est  paramétrable, en fonction du résultat recherché par l'utilisateur. Nous montrons comment cette courbe de facteurs d'actualisation, obtenue à une date donnée, peut être extrapolée au-delà des maturités disponibles sur les marchés, et comment les courbes construites à différentes dates dans le temps, peuvent être utilisées pour obtenir des prévisions de taux d'intérêts dans le futur.  Pour ce dernier point portant sur la prévision, des techniques basées sur l'\textbf{analyse en composantes principales (ACP) fonctionnelle} sont appliquées aux paramètres du modèle. 

Nous présentons ensuite un modèle de prévision simultanée de séries temporelles multivariées, basé sur des réseaux de neurones dits \textbf{Random Vector Functional Link (RVFL) networks}. Les RVFL ont été utilisés avec succès par le passé, pour des problèmes de régression, de classification, et de prévision de séries temporelles univariées. La nouveauté ici, est de les appliquer à des séries temporelles multivariées avec plusieurs contraintes de régularisation sur les paramètres, et des variables explicatives quasi-aléatoires. Les exemples d'application portent sur la prévision de courbes de facteurs d'actualisation. Le modèle ainsi obtenu est un modèle robuste, notamment grâce à ses paramètres de régularisation, et on montre qu'il obtient de très bonnes performances quand il est comparé à des modèles similaires. Il peut également être utilisé dans le cadre de \textit{stress tests}. 

Dans le chapitre 4, des méthodes d'apprentissage dites ensemblistes sont appliquées à la prévision simultanée de taux d'intérêts, et d'autres variables macroéconomiques telles que l'inflation. Ces méthodes ensemblistes sont construites à partir de modèles relativement plus simples; et le modèle de base utilisé ici pour la construction des ensembles est le modèle RVFL. Trois types de méthodes ensemblistes sont envisagées: \textbf{bootstrap aggregating},  \textbf{boosting}, et \textbf{stacked generalization}. On observe ici, généralement, que ces méthodes ensemblistes peuvent permettre d'obtenir des améliorations de performance par rapport aux modèles de base qui les constituent. Il est à noter que cette amélioration ne sera pas observée de façon systématique sur d'autres jeux de données. Mais ce type de modèle aura toujours un intérêt à être envisagé dans une optique d'accroissement des performances prédictives. 

Le chapitre suivant étudie l'application des méthodes d'apprentissage basées sur des noyaux (\textbf{Kernel Regularized Least Squares}, KRLS) à la prévision de courbes de taux d'intérêt. Les modèles de type KRLS expriment la variable à expliquer (ici, le taux \textit{spot}) comme une combinaison linéaire de distances entre les observations des variables explicatives. Ces méthodes sont appliquées ici, d'abord dans le contexte populaire de Nelson-Siegel dynamique qui représente la courbe de taux en fonction de trois facteurs explicatifs, puis directement aux taux \textit{spot}. Cette deuxième approche s'apparentant à des approches appliquées en géostatistique, permet d'obtenir de bonnes performances de prévision, tout en reproduisant des faits stylisés de la courbe de taux sans contrainte de forme supplémentaire.

Le dernier chapitre compare les différents modèles introduits dans les chapitres précédents. Pour ce faire, il revisite le modèle RVFL introduit aux chapitre 3, en adjoignant à ses paramètres une distribution \textit{a priori} (Gaussienne multivariée). Le modèle ainsi obtenu sert d'ingrédient de base à une méthode d'optimisation bayésienne, appliquée à la minimisation de l'erreur de validation croisée des modèles étudiés. De manière générale dans le texte et idéalement, il faudrait introduire en plus de l'erreur de validation croisée, une erreur de validation sur des données nouvelles, non encore rencontrées par les modèles étudiés.   


\section{Motivation and Problem Statement (English version)}
\label{sec:intro:motivation}

The Own Risk Solvency and Assessment (ORSA) is a set of processes defined in the European prudential directive Solvency II, that serve for decision-making and strategic analysis. In the context of ORSA, insurance companies are required to assess their solvency needs in a continuous and prospective way. For this purpose, they notably need to forecast their balance sheet -asset and liabilities- over a defined horizon.

Here, we specifically focus on the assets' forecasting part. \textbf{This thesis is about the Yield Curve, Forecasting, and Forecasting the Yield Curve}. We present a few novel techniques for the construction, the extrapolation of static curves (that is, curves which are constructed at a fixed date), and for forecasting the spot interest rates over time. Cross validation is widely used here as a measure of model performance, but ideally a validation set with unseen data could be considered. 

Throughout the text, when we say 'Yield Curve', we actually mean 'Discount curve'. That is: we ignore the counterparty credit risk, and consider that the curves are \textit{risk-free}. Though, the same techniques could be applied to construct/forecast the actual \textit{risk-free} curves and credit spreads' curves, and combine both to obtain pseudo-discount curves incorporating the counterparty credit risk. The structure of the thesis is described in section \ref{sec:intro:structure}.

\section{Thesis Structure}
\label{sec:intro:structure}

\textbf{Chapter \ref{sec:insurance_swap_curve}} \\[0.2em]

We derived a class of discount curve construction and extrapolation methods,  based on a class of interest rate models called \textit{exogenous short-rate models}. That means: constructing a static Yield Curve at a given date, by using some specific financial instruments with different maturities quoted at this date. Then, defining what are the discount rates beyond the longest maturity observed for these financial instruments. The extrapolated part of the curve is typically necessary for the pricing of long-term insurance liabilities. In the framework that we propose, Yield Curve forecasts can be obtained by using a \textbf{functional principal components analysis} on the model parameters.

\textbf{Chapter \ref{sec:rvfl_mts}} \\[0.2em]

We are interested in obtaining forecasts for multiple time series, by taking into account the potential nonlinear relationships between their observations. For this purpose, we used a specific type of regression model on an augmented dataset of lagged time series. Our model is inspired by dynamic regression models, with the response variable's lags included as predictors, and is known as \textbf{random vector functional link (RVFL) neural networks}. The RVFL neural networks have been successfully applied in the past, to solving regression and classification problems. The novelty of our approach is to apply an RVFL model to multivariate time series, under two separate regularization constraints on the regression parameters, and quasi-randomized features. The model can also be used for realizing \textit{stress tests}. 

\textbf{Chapter \ref{sec:rvfl_ensembles}} \\[0.2em]

The goal of ensemble learning is to combine two or more statistical/machine learning models - the base learners - into one, in order to obtain an ensemble model. The ensemble model is expected to have an improved out-of-sample error over the base models. We apply two popular ensemble learning methods to multiple time series forecasting: \textbf{bootstrap aggregating}, known as bagging, \textbf{boosting}, and \textbf{stacked generalization}, known as stacking. The base learners that we use, are the RVFL introduced in the previous paragraph.


\textbf{Chapter \ref{sec:discount_curve_krls}} \\[0.2em]

\textbf{Kernel regularized least squares} (KRLS) learning methods are applied to Yield Curve forecasting. Two types of formulations of the forecasting problem are tested. One relying on a popular framework called Dynamic Nelson-Siegel, and another one, in which we apply the KRLS directly to the explain the spot rates (response variable) as a function of the observation dates and time to maturities (covariates).

\textbf{Chapter \ref{sec:bayesian_rvfl}} \\[0.2em]

We present a \textbf{bayesian quasi-randomized vector functional link neural network model} (BQRVFL), with one hidden layer. It's a penalized regression model on an augmented data set, in which we assume that a prior multivariate gaussian distribution governs the regression parameters. The BQRVFL model is presented, along with the associated formulas for confidence interval around its predictions. It is then applied as a workhorse for \textbf{bayesian optimization} of machine learning cross-validation functions. The machine learning cross-validation functions that we consider are those associated to the selection of hyperparameters of RVFL (in a DNS framework, or directly applied to the spot rates' time series) and KRLS models (in a DNS framework, or directly applied to the spot rates).

